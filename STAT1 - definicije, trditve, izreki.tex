\documentclass[11pt]{article}
\usepackage[utf8]{inputenc}
\usepackage[slovene]{babel}

\usepackage{amsthm}
\usepackage{amsmath, amssymb, amsfonts}
\usepackage{relsize}
\usepackage{mathrsfs}
\usepackage{bbm}
\usepackage{xcolor}

\newcommand{\R}{\mathbb{R}}
\newcommand{\N}{\mathbb{N}}
\renewcommand{\P}{\mathbb{P}}
\newcommand{\E}{\mathbb{E}}
\renewcommand{\c}{\mathsf{c}}
\newcommand{\set}[1]{\{#1\}}
\newcommand{\oklepaj}[1]{\left(#1\right)}
\newcommand{\1}{\mathbbm{1}}
\newcommand{\rr}{[-\infty,\infty]}
\newcommand{\ra}{\rightarrow}
\newcommand{\5}{\vspace{0.5cm}}
\newcommand{\vp}{(\Omega, \F, \P)}
\newcommand{\ps}{(E, \mathscr{E})}
\newcommand{\bor}{(\R, \B(\R))}

\newcommand{\B}{\mathscr{B}}
\newcommand{\F}{\mathscr{F}}
\newcommand{\PP}{\mathscr{P}}

\theoremstyle{definition}
\newtheorem{definicija}{Definicija}[section]

\theoremstyle{definition}
\newtheorem{trditev}{Trditev}[section]

\theoremstyle{definition}
\newtheorem{izrek}{Izrek}[section]

\theoremstyle{definition}
\newtheorem{metoda}{Metoda}[section]

\newtheorem*{posledica}{Posledica}
\newtheorem*{opomba}{Opomba}
\newtheorem*{komentar}{Komentar}
\newtheorem{lema}{Lema}
\newtheorem*{dokaz}{Dokaz}
\newtheorem*{posplošitev}{Posplošitev}
\newtheorem*{dogovor}{Dogovor}
\newtheorem*{sklep}{Sklep}

\title{Statistika 1 - definicije, trditve in izreki}
\author{Oskar Vavtar \\
po predavanjih profesorja Jaka Smrekarja}
\date{2021/22}

\begin{document}
\maketitle
\pagebreak
\tableofcontents
\pagebreak

% #################################################################################################

\section{Zadostnost in sorodne teme}
\5

\definicija{\textit{Statistični model} je množica dopustnih porazdelitvenih zakonov za slučajni vektor $X$. Označimo jo $\PP$. Zanjo a priori privzamemo, da velja $\P_X \in \PP$. Tu je $\P_X$ porazdelitveni zakon slučajnega vektorja $X$, torej verjetnostna mera definirana s predpisom
$$\P_X(B) ~=~ \P(X \in B)$$
za $B \in \B(\R^n)$. Torej je $\PP$ množica verjetnostnih mer na $(\R^n,\B(\R^n))$.}
\5

\opomba{Če so $X_i$ n.e.p., torej $X_i \overset{\textsc{nep}}{\sim} X_1$, je $\P_X$ produktna verjetnost
\begin{align*}
\P_X ~&=~ \P_{X_1} \times \P_{X_2} \times \ldots \times \P_{X_n} \\
&=~ \P_{X_1} \times \P_{X_1} \times  \ldots \times \P_{X_1},
\end{align*}
in $\PP$ lahko nadomestimo z množico dopustnih porazdelitev za $X_1$.}
\5

\definicija{Model $\PP$ je \textit{parametričen}, če ga je mogoče parametrizirati kot
$$\PP ~=~ \set{\P_\vartheta \mid \vartheta \in \Theta},$$
kjer je $\Theta$ podmnožica nekega $\R^d$ za primerno število $d$.\footnote{Tedaj je vsaka dopustna porazdelitev določena z $d$ realnoštevilskimi parametri.} Običajno na $\Theta$ zahtevamo dodatne pogoje, kot npr. da je diskretna, ali da je odprta ali, splošneje, da je gladka podmnogoterost brez roba. Množici $\Theta$ pravimo \textit{prostor parametrov}. Če model ni parametričen, je \textit{neparametričen}.}
\5

\definicija[?]{Naj bo $\PP = \set{\P_\vartheta \mid \vartheta \in \Theta}$ model, kjer je $\Theta$ neka indeksna množica, in naj bo $\nu$ neka fiksna $\sigma$-končna mera na $(\R^n,\B(\R^n))$. Če za $\forall \vartheta \in \Theta$ velja $\P_\vartheta \ll \nu$, pravimo, da je $\PP$ \textit{dominiran} z $\nu$. Tedaj model označimo z gostotami
$$f(\boldsymbol{\cdot}\,; \vartheta) ~=~ \frac{d\P_\vartheta}{d\nu}.$$
Tedaj velja
$$\P_\vartheta(B) ~=~ \P_\vartheta(X \in B) ~=~ \int_B f(x;\vartheta)\,d\nu(x).$$

% #################################################################################################


\end{document}